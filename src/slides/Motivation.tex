\section*{Motivation}

\begin{frame}{Social-Network Analysis}
    \begin{figure}[h]
        \centering
        \includegraphics[width=0.65\textwidth]{imglib/social-network-analysis}\\
        \label{fig:social-network-analysis}
    \end{figure}
\end{frame}

\begin{frame}{Social-Network Analysis}
    \begin{itemize}
        \item Wer kennt wen?
        \item Wer kennt wen, den ich kenne?
        \item Wer kennt viele, die sich untereinander auch kennen?
        \item Gibt es Gruppen, in denen (fast) jeder jeden kennt?
        \item Wie finde ich besonders eng verbundene Gemeinschaften?
        \item Wie stabil oder robust sind diese Gruppen gegen den Verlust von Verbindungen?
    \end{itemize}
\end{frame}

\begin{frame}{Higher-Order Truss-Decomposition}
    \begin{itemize}
        \item \textbf{Decomposition}: Zerlegung
        \item \textbf{Truss}: Gerüst
        \item \textbf{To truss something}: etwas bündeln
        \item \textbf{Higher-Order}: höherer Ordnung
    \end{itemize}
\end{frame}

\begin{frame}{Higher-Order Truss-Decomposition}
    \begin{itemize}
        \item Analyse von Sub-Graphen ungerichteter Graphen
        \item Bestimmung von Gemeinschaften durch:
        \begin{itemize}
            \item Betrachtung aller Knoten und deren umgebende Knoten
            \item Ermittlung gemeinsamer Nachbarn aller Knoten in einem Verbund
            \item Eliminierung von Kanten in einem Verbund und Überprüfung \textit{Bindung}
        \end{itemize}
        \item Nach: Chen Chen et al., \textit{Parallel Higher-order Truss Decomposition}, 2024 \cite{chen2024parallelhigherordertrussdecomposition}
    \end{itemize}
\end{frame}
